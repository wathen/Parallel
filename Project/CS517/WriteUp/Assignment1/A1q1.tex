

% This LaTeX was auto-generated from an M-file by MATLAB.
% To make changes, update the M-file and republish this document.

\documentclass{article}
\usepackage{graphicx}
\usepackage{color}

\sloppy
\definecolor{lightgray}{gray}{0.5}
\setlength{\parindent}{0pt}

\begin{document}

\section*{Question 1}

    \begin{verbatim}
function A1q1()

clc

% initialising beta, gamma and n for part a
beta = 2;
gamma = -10;
n = 4;

fprintf('\n    ===========================================================')
fprintf('\n      Question 1a) beta = %2.4f, gamma = %2.4f and n = %3i\n',beta,gamma,n)
fprintf('    ===========================================================\n')

% Creating discrete Convection Diffution system
A = ConvectionDiffusion(beta,gamma,n);
Achen = lcd(beta,gamma,n);

% Checking the that matrix A and Achen are the same
fprintf('\n\n      Inf-norm of A-Achen = %1.4e\n',norm(A-Achen,inf));

% Checking that matrix-vector product routine is correct for 10 random
% vectors
fprintf('\n      Inf-norm error between MatVec and inbuild matrix-vector\n')
fprintf('                          product routine:\n\n')
for j = 1:10
    x = rand(n^2,1);
    y = MatVec(A,x);
    fprintf('            Test %2.0i: error = %1.4e\n',j,norm(y-A*x));
end


% initialising beta, gamma and n for part b
beta = 0;
gamma = 0;
n = 10;

fprintf('\n    ===========================================================')
fprintf('\n      Question 1b) beta = %2.0f, gamma = %2.0f and n = %3i\n',beta,gamma,n)
fprintf('    ===========================================================\n')

% producing 2D Laplacian matix
A = ConvectionDiffusion(beta,gamma,n);

% Creating random RHS vector
b = rand(n^2,1);

% Calculating optimal alpah using min and max eigenvalues of the 2D
% Laplacian matrix
minLamda = 4-4*cos(pi/(n+1));
maxLamda = 4-4*cos(n*pi/(n+1));
alpha = 2 /(minLamda+maxLamda);

% Creating initial conditions
x_initial = sparse(n^2,1);
tol = 1e-6;

% setting parameter to display results at every Mth iteration
M = 20;

% Calculating spectral radius of iteration matrix and appox number of iters
spectrad = (maxLamda-minLamda)/(minLamda+maxLamda);
fprintf('\n\n     Spectral radius of iteration matrix = %1.4f ',spectrad)
fprintf('\n     Approximate number of iterations = %4.0f \n\n\n',log10(tol)...
/log10(0.9595))


% Calling richardson routine
x = richardson(A,b,alpha,x_initial, tol,M);

% Checking that Richardson routine calculates the right answer
fprintf('\n     Inf-norm error between Richardson approx and backslash = %1.4e',norm(x-A\b,inf))


    function [A] = ConvectionDiffusion(beta,gamma,n)

        e = ones(n,1);

        % Creating sparse diagonal matrices
        I = spdiags(e,0,n,n);
        I1 =spdiags(e,1,n,n);
        I2 = spdiags(e,-1,n,n);

        % Creating 1D Convection-Diffusion matricies
        Abeta = 2*I +(beta-1)*I1 - (beta+1)*I2;
        Agamma = 2*I +(gamma-1)*I1 - (gamma+1)*I2;

        % Creating 2D Convection-Diffusion matrix
        A = kron(I,Abeta)+kron(Agamma,I);

    end

    function A=lcd(beta,gamma,n)

        %
        % lcd.m: Laplacian-Convection-Diffusion, simple code to generate the matrices
        %        for assignment 1
        % Usage:
        %   - to generate the Laplacian (symmetric positive definite), call with
        %   beta=gamma=0: A=lcd(0,0,n);
        % - to generate a convection-diffusion matrix (nonsymmetric) call, e.g.,
        % with beta=0.5, gamma=0.6: A=lcd(0.5,0.6,n);
        %
        % Note: output matrix is of size n^2-by-n^2
        %

        ee=ones(n,1);
        a=4; b=-1-gamma; c=-1-beta; d=-1+beta; e=-1+gamma;
        t1=spdiags([c*ee,a*ee,d*ee],-1:1,n,n);
        t2=spdiags([b*ee,zeros(n,1),e*ee],-1:1,n,n);
        A=kron(speye(n),t1)+kron(t2,speye(n));
    end

    function y = MatVec(A,x)

        % finding sparsity partern of A
        [ii,jj,~] = find(A);

        y = spalloc(length(x),1,length(x));

        % looping through nonzero components of A to calculate the matrix
        % vector multiplication
        for i = 1:length(ii)
            y(ii(i)) = y(ii(i))+A(ii(i),jj(i))*x(jj(i));
        end

    end

    function x_out = richardson(A,b,alpha,x_initial, tol,M)

        % Creating fprintf logs
        titlelog = '\n\n\n        %4s   |     %5s       |    %5s \n';
        iterlog = '        %4i    |    %1.4e    |     %1.6f   \n';

        % Using sparse MatVec routine to calulation A*x_intial
        r = b-MatVec(A,x_initial);
        % Since x_inital is a vector of zeros then r_norm = b_norm but here
        % I have writen a general Richardson scheme
        r_norm = norm(r);
        b_norm = norm(b);
        ii = 1;
        xx = x_initial;
        fprintf(titlelog,'Iter','RelRes','Conver rate')
        fprintf('      ----------------------------------------------\n')

        fprintf(iterlog,0,r_norm/b_norm, 0)
        check = r_norm/b_norm;
        while check > tol
            % Storing old residual
            resold = r_norm/b_norm;
            % updating xx
            xx = xx +alpha*r;
            % Recalculating residual
            r = b-MatVec(A,xx);
            r_norm = norm(r);
            % printing results
            check = r_norm/b_norm;
            if mod(ii,M) == 0 || check < tol
                fprintf(iterlog,ii,r_norm/b_norm,1/(resold*b_norm/r_norm))
            end
            ii = ii + 1;
        end
        x_out = xx;
    end

end
\end{verbatim}

        \color{lightgray} \begin{verbatim}
    ===========================================================
      Question 1a) beta = 2.0000, gamma = -10.0000 and n =   4
    ===========================================================

      Inf-norm of A-Achen = 0.0000e+00

      Inf-norm error between MatVec and inbuild matrix-vector
                          product routine:

            Test  1: error = 0.0000e+00
            Test  2: error = 0.0000e+00
            Test  3: error = 0.0000e+00
            Test  4: error = 0.0000e+00
            Test  5: error = 0.0000e+00
            Test  6: error = 0.0000e+00
            Test  7: error = 0.0000e+00
            Test  8: error = 0.0000e+00
            Test  9: error = 0.0000e+00
            Test 10: error = 0.0000e+00

    ===========================================================
      Question 1b) beta =  0, gamma =  0 and n =  10
    ===========================================================

     Spectral radius of iteration matrix = 0.9595
     Approximate number of iterations =  334

        Iter   |     RelRes       |    Conver rate
      ----------------------------------------------
           0    |    1.0000e+00    |     0.000000
          20    |    3.3910e-01    |     0.959423
          40    |    1.4826e-01    |     0.959490
          60    |    6.4840e-02    |     0.959493
          80    |    2.8358e-02    |     0.959493
         100    |    1.2403e-02    |     0.959493
         120    |    5.4244e-03    |     0.959493
         140    |    2.3724e-03    |     0.959493
         160    |    1.0376e-03    |     0.959493
         180    |    4.5379e-04    |     0.959493
         200    |    1.9847e-04    |     0.959493
         220    |    8.6802e-05    |     0.959493
         240    |    3.7963e-05    |     0.959493
         260    |    1.6604e-05    |     0.959493
         280    |    7.2617e-06    |     0.959493
         300    |    3.1760e-06    |     0.959493
         320    |    1.3890e-06    |     0.959493
         328    |    9.9780e-07    |     0.959493

     Inf-norm error between Richardson approx and backslash = 6.1753e-06\end{verbatim} \color{black}





We expect to see that the rate of convergence should be the same as the spectral radius of the iteration matrix. We can see that this is indeed the case from the table above. Using the formula
$$k \approx \frac{-m}{\log_{10}\rho},    $$
where $k$, $m$ and $\rho$ are the number of iteration, $-\log_{10} tol$ and the spectral radius respectively we see that our approximation for the number of iterations is pretty accurate.

\end{document}





