\documentclass[10pt]{article}


\usepackage{amsmath}
\usepackage{amsfonts}
\usepackage{graphicx}
\usepackage{nicefrac}
\usepackage{subfigure}
\usepackage{algorithm}
\usepackage{paralist}
\usepackage[geometry]{ifsym}
\usepackage{rotating}
\usepackage[normalem]{ulem}
\usepackage{cite}
\usepackage{nicefrac}
\usepackage{algpseudocode}
\usepackage{varwidth}

\bibliographystyle{amsplain}

\addtolength{\oddsidemargin}{-.2in}
    \addtolength{\evensidemargin}{-.5in}
    \addtolength{\textwidth}{0.5in}

    \addtolength{\topmargin}{-0.5in}
    \addtolength{\textheight}{0.35in}

\sloppy                 % makes TeX less fussy about line breaking

\pagestyle{plain}           % use just a plain page number

\numberwithin{equation}{section}    % add the section number to the equation label



\usepackage{fancyheadings}

\newcommand{\com}[1]{\texttt{#1}}
\newcommand{\DIV}{\ensuremath{\mathop{\mathbf{DIV}}}}
\newcommand{\GRAD}{\ensuremath{\mathop{\mathbf{GRAD}}}}
\newcommand{\CURL}{\ensuremath{\mathop{\mathbf{CURL}}}}
\newcommand{\CURLt}{\ensuremath{\mathop{\overline{\mathbf{CURL}}}}}
\newcommand{\nullspace}{\ensuremath{\mathop{\mathrm{null}}}}


\newcommand{\FrameboxA}[2][]{#2}
\newcommand{\Framebox}[1][]{\FrameboxA}
\newcommand{\Fbox}[1]{#1}

%\usepackage[round]{natbib}

\newcommand{\half}{\mbox{\small \(\frac{1}{2}\)}}
\newcommand{\hf}{{\frac 12}}
\newcommand {\HH}  { {\bf H} }
\newcommand{\hH}{\widehat{H}}
\newcommand{\hL}{\widehat{L}}
\newcommand{\bmath}[1]{\mbox{\bf #1}}
\newcommand{\hhat}[1]{\stackrel{\scriptstyle \wedge}{#1}}
\newcommand{\R}{{\rm I\!R}}
\newcommand {\D} {{\vec{D}}}
\newcommand {\sg}{{\hsigma}}
%\renewcommand{\vec}[1]{\ensuremath{\mathbf{#1}}}
\newcommand{\E}{\vec{E}}
\renewcommand{\H}{\vec{H}}
\newcommand{\J}{\vec{J}}
\newcommand{\dd}{d^{\rm obs}}
\newcommand{\F}{\vec{F}}
\newcommand{\C}{\vec{C}}
\newcommand{\s}{\vec{s}}
\newcommand{\N}{\vec{N}}
\newcommand{\M}{\vec{M}}
\newcommand{\A}{\vec{A}}
\newcommand{\B}{\vec{B}}
\newcommand{\w}{\vec{w}}
\newcommand{\nn}{\vec{n}}
\newcommand{\cA}{{\cal A}}
\newcommand{\cQ}{{\cal Q}}
\newcommand{\cR}{{\cal R}}
\newcommand{\cG}{{\cal G}}
\newcommand{\cW}{{\cal W}}
\newcommand{\hsig}{\hat \sigma}
\newcommand{\hJ}{\hat \J}
\newcommand{\hbeta}{\widehat \beta}
\newcommand{\lam}{\lambda}
\newcommand{\dt}{\delta t}
\newcommand{\kp}{\kappa}
\newcommand {\lag} { {\cal L}}
\newcommand{\zero}{\vec{0}}
\newcommand{\Hr}{H_{red}}
\newcommand{\Mr}{M_{red}}
\newcommand{\mr}{m_{ref}}
\newcommand{\thet}{\ensuremath{\mbox{\boldmath $\theta$}}}
\newcommand{\curl}{\ensuremath{\nabla\times\,}}
\renewcommand{\div}{\nabla\cdot\,}
\newcommand{\grad}{\ensuremath{\nabla}}
\newcommand{\dm}{\delta m}
\newcommand{\gradh}{\ensuremath{\nabla}_h}
\newcommand{\divh}{\nabla_h\cdot\,}
\newcommand{\curlh}{\ensuremath{\nabla_h\times\,}}
\newcommand{\curlht}{\ensuremath{\nabla_h^T\times\,}}
\newcommand{\Q}{\vec{Q}}
\renewcommand{\J}{\vec J}
\renewcommand{\J}{\vec J}
\newcommand{\U}{\vec u}
\newcommand{\Bt}{B^{\mbox{\tiny{T}}}}
\newcommand{\me}{Maxwell's equations }
\newcommand{\ns}{Navier-Stokes Equations }
\renewcommand{\s}{Stokes Equations }
\newcommand{\Fs}{\vec{f}_{\mbox{\tiny s}}}
\newcommand{\partialt}[1]{\frac{\partial #1}{\partial t}}
\newcommand{\cref}[1]{(\ref{#1})}
% \newcommand{\Ct}{\ensuremath{C^{\mbox{\tiny{T}}}}
\newcommand{\Ct}{\ensuremath{C^{\mbox{\tiny{T}}}}}
% \renewcommand{\baselinestretch}{1.40}\normalsize
\usepackage{setspace}
\usepackage{amsthm}
\newtheorem{prop}{Proposition}[section]

\onehalfspacing
\begin{document}
\pagestyle{fancyplain}
\fancyhead{}
\fancyfoot{} % clear all footer fields
\fancyfoot[LE,RO]{\thepage \hspace{-5mm}}
\fancyfoot[LO,CE]{ \footnotesize{ Michael Wathen 7830121}}
\fancyfoot[CO,RE]{}

\title{CS517 - Project proposal}
\author{Michael Wathen}
\maketitle

In this project we will be considering the solution of a divergence free finite element approximation to the steady state Stokes equations:
\begin{equation} \label{eq:Stokes}
    \begin{aligned}
        -\nu \Delta \U- \nabla p &= \vec f,\\
        \div \U     &= 0,
    \end{aligned}
\end{equation}
where $\U$ is the velocity vector, $p$ is the pressure and $\nu$ is the dynamic viscosity. Using a mixed discretisation we get the following saddle point system
$$\begin{bmatrix}
A & \Bt\\
B & 0\\
\end{bmatrix} \begin{bmatrix}
u \\
p \\
\end{bmatrix} = \ \begin{bmatrix}
f \\
0 \\
\end{bmatrix},$$
where $A$, $\Bt$ and $B$ are the discrete laplacian, gradient and divergence matrices. First we will go through what makes a good preconditioner and then look at a good approximation to ideal Schur complement preconditioner. Using the preconditioner proposed in \cite{Wathen1993,Silvester1994} namely
$$\mathcal{P} = \begin{bmatrix}
A & 0\\
0 & M\\
\end{bmatrix},$$
where $M$ is the mass matrix, we will show the important property that $M$ is spectrally equivalent to the Schur complement.

Using a multigrid V-cycle  (see \cite{Trottenberg2000,Briggs2000} for description) to the (1,1) block and Conjugate Gradient \cite{Hestenes1952} with a Jacobi preconditioner with the mass works extremely well for standard Taylor-Hood ($\mathbb{P}_2/\mathbb{P}_1$) elements. However, applying a standard V-cycle to the (1,1) block when using the divergence free elements ($\mathbb{BDM}_2/\mathbb{DG}_1$) will cause the number of outer iterations of the iterative scheme to increase. Therefore, we will be looking at an algebraic multigrid methods (AMG) approach where one can change the parameters (which maybe more suitable for these elements) within the the V-cycle.

\bibliographystyle{apalike}
\bibliography{ref}




\end{document}
