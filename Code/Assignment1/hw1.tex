 %   Journal Article Template
%%%%%%%%%%%%%%%%%%%%%%%%%%%%%%%%%%%%%%%%%%%%%%%%%%%%%%%%%%%%%%%%%%%%%%%%
\documentclass{article}
\usepackage{graphicx}
\usepackage{amssymb}
\usepackage{bm}
%\usepackage[notref,notcite]{showkeys}
%\usepackage[dvipdfm]{hyperref}
%\usepackage{hyperref}
\usepackage{graphicx}
%\usepackage{subfigure}
%\usepackage{epsfig,psfig}
\usepackage{amsfonts,amsmath,latexsym}
\usepackage{amsbsy}
\usepackage{nicefrac}
\usepackage{subcaption}
\usepackage{color}
\usepackage[usenames,dvipsnames]{xcolor}
\usepackage{sidecap}
\usepackage{calc}
\usepackage{enumitem}
\usepackage{listings}
%%%%%%%%%%%%%%%%%%%%%%%%%%%%%%%%%%%%%%%%%%%%%%%%%%%%%%%%%s
%%%%%%%%%%%%%%%%%%%%%%%%%%%%%%%%%%%%%%%%%%%%%%%%%%%%%%%%%%%


\begin{document}

\title{CS521 - Assignment 1}
\author{Michael Wathen}
\maketitle

\paragraph{Collaborators} ~\\
Ehsan Kermani helped me with the eunit test file as well as programming the  {\tt{hw1:subtract}} function. Mark Greenstreet helped me understand and implement the {\tt{hw1:search}} function.



\vspace{1cm}


\section*{Question 1:}

To run the tests
\begin{lstlisting}
c(hw1).
c(hw1_test).
eunit:test(hw1_test).
\end{lstlisting}
\subsection*{Part a}

See {\tt nthtail} function given in file hw1.erl



\subsection*{Part b}

{\tt prefix} function given in file hw1.erl


\subsection*{Part c}



See {\tt prefix} function given in file hw1.erl


\section*{Question 2}

\subsection*{Part a}

See {\tt libSubtraction} and {\tt subPrint} functions given in file hw1.erl

\noindent See Table~\ref{tab:results} for the results.




\subsection*{Part b}

From the results in Table~\ref{tab:results}  we see that the values in the time/$(N^2)$ column in the table staysroughly constant. Therefore, the {\tt lists:subtract} library function appears to be quadratic with respect to run time.

Without looking at the source code for {\tt lists:subtract}, this behavior would be expected if every element of {\tt L1} is checked by every element in {\tt L2}. This would account for $N^2$ operations.

\subsection*{Part c}

See {\tt subtract} and {\tt elementCheck} functions given in file hw1.erl

\subsection*{Part d}

The results of my {\tt subtract} function are given in Table~\ref{tab:results}. The timing results seem to be roughly $\mathcal{O}(N \log N)$ from columns time/$(N\log N)$ and time/$N$. We achieve this speed up in the {\tt subtract} function since we order both lists to start with. This enables us to go through them together for comparison.

\begin{table}
\centering
\begin{tabular}{|c|cc|ccc|}
\hline
\hline\\[-0.35cm]
  $N$ & \tt{lists:subtract} & time/$(N^2)$& \tt{hw1:subtract} & time/$(N\log N)$& time/$N$\\[0.05cm]
\hline
\hline

1000 & 17945 & 1.79e-02 & 128 & 1.85e-02 & 1.28e-01 \\
2000 & 7446 & 1.86e-03 & 315 & 2.07e-02 & 1.58e-01 \\
3000 & 17203 & 1.91e-03 & 537 & 2.24e-02 & 1.79e-01 \\
5000 & 70788 & 2.83e-03 & 672 & 1.58e-02 & 1.34e-01 \\
10000 & 203369 & 2.03e-03 & 1507 & 1.64e-02 & 1.51e-01 \\
20000 & 721388 & 1.80e-03 & 2899 & 1.46e-02 & 1.45e-01 \\
30000 & 1593127 & 1.77e-03 & 5146 & 1.66e-02 & 1.72e-01 \\
50000 & 4423758 & 1.77e-03 & 6631 & 1.23e-02 & 1.33e-01 \\
\hline
\hline
\end{tabular}
\caption{Timing table for question 2. {\tt lists:subtract} and {\tt hw1:subtract} denote the elapsed time with the library function and my own (2c), respectively}
\label{tab:results}
\end{table}


\vspace{1cm}

I'm working hard to try and understand {\tt erlang} and functional programming so I apologise for extra length and poor {\tt erlang} programming customs in this assignment.

\end{document}




